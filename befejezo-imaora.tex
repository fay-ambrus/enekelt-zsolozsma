\documentclass{article}
\usepackage{gregosheet}
\usepackage{gregosheet-psalm}
% Font sizes
\newcommand{\musicfontsize}{20}
\newcommand{\lyricfontsize}{10}
\newcommand{\psalmfontsize}{10}

% Fonts
% Change this to any font command you want:
% \rmfamily (roman/serif), \sffamily (sans-serif), \ttfamily (monospace)
% or use \setmainfont{FontName} or \newfontfamily for specific fonts
\newfontfamily\lyricfont{Cambria}
\newfontfamily\psalmfont[BoldFont={Cambria Bold}]{Cambria}
\newfontfamily\headingfont[BoldFont={Cambria Bold}]{Cambria}
\setmainfont{Cambria}

% Spacing
\newcommand{\systemvskip}{10pt}
\newcommand{\lyricvskip}{2pt}
\newcommand{\blockvskip}{10pt}

% Page layout
\usepackage[a5paper,margin=0.5in]{geometry}
\pagestyle{empty}
\setlength{\parindent}{0pt}

% Title style
\makeatletter
\renewcommand{\maketitle}{
  \begin{center}
    \headingfont\fontsize{24}{28}\selectfont\bfseries\MakeUppercase{\@title}
    \vskip 1em
    \headingfont\fontsize{13}{16}\selectfont\MakeUppercase{\@subtitle}
  \end{center}
}
\newcommand{\subtitle}[1]{\gdef\@subtitle{#1}}
\newcommand{\@subtitle}{}
\makeatother

% Section style
\usepackage{titlesec}
\titleformat{\section}
  {\centering\headingfont\fontsize{20}{24}\selectfont\bfseries}
  {}{0pt}{\MakeUppercase}
\titlespacing*{\section}{0pt}{\blockvskip}{\blockvskip}

% Subsection style
\titleformat{\subsection}
  {\centering\headingfont\fontsize{16}{20}\selectfont\bfseries\addfontfeature{LetterSpace=10.0}}
  {}{0pt}{\MakeUppercase}
\titlespacing*{\subsection}{0pt}{\blockvskip}{\blockvskip}

% Subsubsection style
\titleformat{\subsubsection}
  {\centering\headingfont\fontsize{12}{14}\selectfont\bfseries\scshape}
  {}{0pt}{}
\titlespacing*{\subsubsection}{0pt}{\blockvskip}{\blockvskip}

% Part title command
\newcommand{\parttitle}[2][]{
  \vskip\blockvskip
  \par\noindent
  \headingfont\fontsize{10}{12}\selectfont
  \textcolor{red}{\MakeUppercase{#2}}%
  \ifx&#1&\else\hfill\textcolor{red}{#1}\fi
  \par
}

% Commentary command
\newcommand{\commentary}[1]{
  \par\noindent
  {\fontsize{8}{10}\selectfont\textcolor{red}{#1}}
  \par
}

% Red text command
\newcommand{\redtext}[1]{
  \par\noindent
  \textcolor{red}{#1}
  \par
}

% Red dash command
\newcommand{\reddash}{\textcolor{red}{\textemdash{}}\xspace}
\usepackage{xspace}

% Lectio title command
\newcommand{\lectiotitle}[1]{
  \par\noindent
  {\centering\textcolor{red}{\textit{#1}}\par}
  \setlength{\parindent}{0.2in}
  \everypar{\setlength{\parindent}{0.2in}}
}

% Versicle-Response command
\newcommand{\versicle}[2]{
  \par\noindent
  \hangindent=1em\hangafter=1
  \emergencystretch=1em
  \makebox[1em][l]{\textcolor{red}{V.}}#1
  \par\noindent
  \hangindent=1em\hangafter=1
  \emergencystretch=1em
  \makebox[1em][l]{\textcolor{red}{F.}}#2
  \par
}

% Hymn stanzas environment
\usepackage{multicol}
\newenvironment{hymnstanzas}{
  \setlength{\leftmargin}{0pt}
  \setlength{\rightmargin}{0pt}
  \setlength{\columnsep}{4em}
  \begin{multicols}{2}
  \raggedright
}{
  \end{multicols}
}

\newcommand{\stanzabreak}{%
  \par\addvspace{0.8\baselineskip}%
}


% Single centered stanza (for odd last stanza)
\newcommand{\centerstanza}[1]{
  \par
  \begin{center}
  \begin{minipage}{0.4\textwidth}
  \raggedright
  #1
  \end{minipage}
  \end{center}
  \par
}

% Structured text environment
\newenvironment{preces}[2]{
  \par\noindent
  #1
  \par\noindent
  \hskip0.2in\textit{#2}
  \par
  \setlength{\parindent}{0pt}
  \everypar{\hangindent=0.2in\hangafter=1}
}{
  \par
}

\usepackage{ragged2e}


\title{Befejező imaóra}
\subtitle{Gregorián énekekkel}

\begin{document}

\maketitle

\section*{Általános rész}

\parttitle{Imaóra kezdete}
\versicle{Istenem, jöjj segítség\underline{e}mre!}{%
  Uram, segíts meg \underline{e}ngem! Dicsőség az Atyának a Fiúnak és a Szentlél\underline{e}knek, %
  miképpen kezdetben, most és mindörökké. \underline{A}men. All\underline{e}l\underline{u}ja.%
}
\commentary{Nagyböjtben azonban az Alleluja elmarad.}

\parttitle{Lelkiismeret-vizsgálat}
\commentary{%
  Dicséretes dolog most a lelkiismeret-vizsgálat, amely közös végzés esetén a mise bűnbánati %
  formája szerint történhet.%
}

\vskip0.9\blockvskip

Gyónom a mindenható Istennek és nektek, testvéreim, hogy sokszor és sokat vétkeztem gondolattal,
szóval, cselekedettel és mulasztással:

\commentary{és mindnyájan mellüket verve mondják:}

én vétkem, én vétkem, én igen nagy vétkem.

\commentary{Tovább folytatják:}

Kérem ezért a Boldogságos, mindenkor Szűz Máriát, az összes angyalokat és szenteket, és titeket,
testvéreim, hogy imádkozzatok érettem Urunkhoz, Istenünkhöz.

\vskip0.8\blockvskip

\commentary{Vagy:}

\vskip0.8\blockvskip

\versicle{Urunk, Istenünk, könyörülj rajtunk.}{Mert vétkeztünk ellened.}

\versicle{Urunk, mutasd meg nekünk irgalmasságodat.}{És add meg nekünk az üdvösséget.}

\vskip0.8\blockvskip

\commentary{Vagy:}

\vskip0.8\blockvskip

\versicle{%
  Jézus Krisztus, téged elküldött az Atya, hogy gyógyítsd a töredelmes szívűeket: Uram, irgalmazz!%
}{Uram, irgalmazz!}

\versicle{Jézus Krisztus, te eljöttél, hogy magadhoz hívd a bűnösöket: Krisztus, kegyelmezz!}
  {Krisztus, kegyelmezz!}

\versicle{Jézus Krisztus, te az Atya jobbján ülsz, hogy közbenjárj értünk: Uram, irgalmazz!}
  {Uram, irgalmazz!}

\vskip0.8\blockvskip

\commentary{Végül:}

\vskip0.8\blockvskip

\versicle{%
  Irgalmazzon nekünk a mindenható Isten, bocsássa meg bűneinket, és vezessen el az örök életre.%
}{Ámen.}

\parttitle{Himnusz}

\commentary{%
  Az alábbi két himnusz közül választhatunk, kivéve a következőkben felsorolt napokat. Adventben %
  december 16-áig, karácsonyi időben Vízkereszt főünnepéig, Nagyböjt I., III. és V. hetében, %
  Nagyhéten Nagycsütörtökig az \textcolor{black}{Immár a nap leáldozott…} kezdetű himnuszt kell %
  énekelni. Adventben december 16. után, Vízkereszt főünnepén és Nagyböjt II. és IV. hetében a %
  \textcolor{black}{Krisztus tündöklő nappalunk…} kezdetű himnuszt kell énekelni. Húsvéti időben a %
  később közölt \textcolor{black}{Világmegváltó Jézusunk…} kezdetű himnuszt kell énekelni.%
}

% Hymnus: Te lucis ante terminum
\begin{gregosheet}
  \melody{<-7--7---7-7---7--7--6--5---:-7--7---7-----7--4----5--4--3--,-5----5--6----7--6--5----4--5zZ5-:-5-5---4----3--4----5--4--4-.}
  \lyrics{  Im-már a nap le-ál-do-zott, Te-rem-tőnk, ké-rünk té-ge-det, légy ke-gyes és ma-radj ve-lünk,  ő-riz-zed, óv-jad  né-pe-det!}
\end{gregosheet}

\begin{hymnstanzas}
  Álmodjék rólad a szívünk, \\
  álmunkban is te légy velünk, \\
  téged dicsérjen énekünk, \\
  midőn új napra ébredünk!
\stanzabreak
  Adj nekünk üdvös életet, \\
  szítsd fel a szív fáradt tüzét, \\
  a ránk törő rút éjhomályt \\
  világosságod rontsa szét!
\end{hymnstanzas}
\centerstanza{%
  Kérünk, mindenható Atyánk, \\
  Úr Jézus Krisztus érdemén, \\
  ki Szentlélekkel és veled \\
  uralkodik, s örökre él. Ámen.
}

\commentary{Vagy más himnusz ugyanazon a dallom:}

% Hymnus: Christe qui splendor
\begin{hymnstanzas}
  Krisztus, tündöklő nappalunk, \\
  az éj árnyát eloszlatod, \\
  kit fényből fénynek vall a hit \\
  és fénnyel áldod szentjeid. \\
\stanzabreak
  Esengünk hozzád, szent Urunk, \\
  míg tart az éj, vigyázz reánk, \\
  találjunk benned pihenést, \\
  adj nyugodalmas éjszakát. \\
\stanzabreak
  Mikor lecsukjuk két szemünk, \\
  a szívünk virrasszon veled, \\
  őrködjék mindig jobb kezed \\
  hűséges híveid felett. \\
\stanzabreak
  Védelmezőnk, tekints le ránk, \\
  igázd le ellenségeink, \\
  vezesd szolgáidat, kikért \\
  véred váltsága volt a bér. \\
\end{hymnstanzas}
\centerstanza{%
  Legyen most néked, Krisztusunk, \\
  és szent Atyádnak tisztelet, \\
  s a Szentléleknek is veled \\
  zengjen dicséret szüntelen. Ámen. \\
}

\commentary{%
  Vagy húsvéti időben a \textcolor{black}{Szentlélek Isten szállj le ránk...} kezdetű ének %
  dallamára:
}

% Hymnus: Iesu redemptor saculi
\begin{hymnstanzas}
  Világmegváltó Jézusunk, \\
  örök fönségben szült Ige, \\
  a titkos fényből fénysugár, \\
  népednek őrző pásztora. \\
\stanzabreak
  Mindent hatalmad alkotott, \\
  törvényed rendelt éjt, napot, \\
  a munkás, fáradt testeket \\
  az éj csendjén pihenteted. \\
\stanzabreak
  Poklok hatalmát Megtörő, \\
  légy ellenségtől őrizőnk, \\
  ne tudja bűnbe rántani, \\
  kiket szent véred újraszült. \\
\stanzabreak
  Amíg mulandó életünk \\
  e földi testhez kötve tart: \\
  testünk ha álom nyomja el, \\
  lelkünk virrasszon éberen.
\end{hymnstanzas}
\centerstanza{%
  Dicsérünk, Jézus, szüntelen,
  ki feltámadtál győztesen,
  Atyának, Léleknek veled
  most és örökre tisztelet. Ámen.
}

\parttitle{Rövid válaszos ének}

\begin{gregosheet}
  \melody{<X-rR3-4t-,-4--4---4--3-4t--4---ed1-3r-3-,,-5---5---5z--5---5---4---4t-4-:--5--4--4---rR3-4-4t-,,-5--5---5---5--5-5z--5----5-4--4t-4-:-5--4-4-----rR3-4---4t-.}
  \lyrics{     U-ram, ke-zed-be a-ján-lom lel-ke-met. Meg-vál-tot-tál min-ket U-runk, hű-sé-ges Is-te-nünk. Di-cső-ség az A-tyá nak, a Fi-ú-nak, és a Szent-lé-lek- nek.}
\end{gregosheet}

\commentary{Húsvét nyolcadában a Válaszos ének helyett a következő antifónát énekeljük:}

\begin{gregosheet}
  \melody{<X-3r-3-,-4--4---4--3-4t--4---ed1-3r-3-,,-5---5---5z--5---5---4---4t-4-:--5--4--4---rR3-4-4t-,,-5--5---5---5--5-5z--5----5-4--4t-4-:-5--4-4-----rR3-4---4t-.}
  \lyrics{     U-ram, ke-zed-be a-ján-lom lel-ke-met. Meg-vál-tot-tál min-ket U-runk, hű-sé-ges Is-te-nünk. Di-cső-ség az A-tyá nak, a Fi-ú-nak, és a Szent-lé-lek- nek.}
\end{gregosheet}

%Nagycsütörtökön a Válaszos ének helyett a következő antifónát énekeljük:
%MX-33r---3-:--¢--------------4tg3E2-11rf2-33r---3-:---3t-34t77-iI7i™uU6u-55-44-:
%         Krisz- tus     engedelmes volt ér            -             tünk mind
%MX-4---5---7uj5g3-zh4tT4t-3rR3---3-.
%           a     ha -  lá              -               lig.
%Nagypénteken és Nagyszombaton (ha tartunk Befejező imaórát,) a Válaszos ének helyett a következő antifónát énekeljük:
%MX-33r---3-:--¢--------------4tg3E2-11rf2-33r---3-:---3t-34t77-iI7i™uU6u-55-44-:
%         Krisz- tus     engedelmes volt ér            -             tünk mind
%MX-4---5---7uj5g3-zh4tT4t-3rR3---3-:--3---3---5z----445z-uU6h44t-zZ5T4---3---3---3
%           a     ha -  lá              -               lig,    a     ke - reszt-nek                             ha - lá -  lá-
%MX-eD0-3r-tT4R3E24t3tT4t-rR3-.
%           ig.
%
%M-3---3---4---3-:--3---3---3---3---3---2---4---5-,--5---tT4---rR3---3-:
%       Ez    az     a   nap,  me-lyet  az    Úr szer-zett ne-künk, ör-ven - dez -  zünk,
% 
%M-1---3---4----rR3---3-.
%     s ví- gad- junk  raj -  ta.
%Húsvéti időben Húsvét nyolcadán kívül:

\begin{gregosheet}
  \melody{<-4--5---rR3--4---4t---4-4--,-5--7----6u---5---5--5----rR3-4t-4-.}
  \lyrics{  Kö-nyö-rülj raj-tam, U-ram, és hall-gasd meg kö-nyör-gé- se-met!}
\end{gregosheet}

\begin{psalmsheet}[tone=8, number={4. zsoltár}, title={Hálaadás}, motto={Az Úr csodálatra méltóvá tette azt, akit feltámasztott a holtak közül (Szt. Ágoston).}]
  \psalmtext{
    Halld meg kiáltásomat, igazságos Istenem, * szorult helyzetemben segíts meg engem. \\
    Könyörülj rajtam, * és hallgasd meg imámat! ---

    Meddig lesz még kemény a szívetek, emberek? * Miért szeretitek a haszontalanságot, meddig törekedtek még hazugságra? \\
    Tudjátok meg: az Úr csodásan megmenti hívét; * meghallgat az Úr, ha hozzá kiáltok. ---

    Bár haragusztok, többé ne vétkezzetek! † Szálljatok magatokba szívből, * nyugovóra térve teremtsetek csendet. \\
    Igaz áldozatot hozzatok, * és bízzatok az Úrban! ---

    Azt mondják sokan: „Ki részesít jóban minket?” * Ragyogtasd ránk, Uram, arcod fényességét! \\
    Szívembe nagyobb örömet öntöttél, * mint azoknak, akik dúskálnak búzában meg borban! \\
    Alig térek nyugovóra, békében elalszom, * mert biztonságot egyedül te adsz nekem, Uram.
}
\end{psalmsheet}

\end{document}
