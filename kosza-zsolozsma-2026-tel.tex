\documentclass{article}
\usepackage{gregosheet}
\usepackage{gregosheet-psalm}
% Font sizes
\newcommand{\musicfontsize}{20}
\newcommand{\lyricfontsize}{10}
\newcommand{\psalmfontsize}{10}

% Fonts
% Change this to any font command you want:
% \rmfamily (roman/serif), \sffamily (sans-serif), \ttfamily (monospace)
% or use \setmainfont{FontName} or \newfontfamily for specific fonts
\newfontfamily\lyricfont{Cambria}
\newfontfamily\psalmfont[BoldFont={Cambria Bold}]{Cambria}
\newfontfamily\headingfont[BoldFont={Cambria Bold}]{Cambria}
\setmainfont{Cambria}

% Spacing
\newcommand{\systemvskip}{10pt}
\newcommand{\lyricvskip}{2pt}
\newcommand{\blockvskip}{10pt}

% Page layout
\usepackage[a5paper,margin=0.5in]{geometry}
\pagestyle{empty}
\setlength{\parindent}{0pt}

% Title style
\makeatletter
\renewcommand{\maketitle}{
  \begin{center}
    \headingfont\fontsize{24}{28}\selectfont\bfseries\MakeUppercase{\@title}
    \vskip 1em
    \headingfont\fontsize{13}{16}\selectfont\MakeUppercase{\@subtitle}
  \end{center}
}
\newcommand{\subtitle}[1]{\gdef\@subtitle{#1}}
\newcommand{\@subtitle}{}
\makeatother

% Section style
\usepackage{titlesec}
\titleformat{\section}
  {\centering\headingfont\fontsize{20}{24}\selectfont\bfseries}
  {}{0pt}{\MakeUppercase}
\titlespacing*{\section}{0pt}{\blockvskip}{\blockvskip}

% Subsection style
\titleformat{\subsection}
  {\centering\headingfont\fontsize{16}{20}\selectfont\bfseries\addfontfeature{LetterSpace=10.0}}
  {}{0pt}{\MakeUppercase}
\titlespacing*{\subsection}{0pt}{\blockvskip}{\blockvskip}

% Subsubsection style
\titleformat{\subsubsection}
  {\centering\headingfont\fontsize{12}{14}\selectfont\bfseries\scshape}
  {}{0pt}{}
\titlespacing*{\subsubsection}{0pt}{\blockvskip}{\blockvskip}

% Part title command
\newcommand{\parttitle}[2][]{
  \vskip\blockvskip
  \par\noindent
  \headingfont\fontsize{10}{12}\selectfont
  \textcolor{red}{\MakeUppercase{#2}}%
  \ifx&#1&\else\hfill\textcolor{red}{#1}\fi
  \par
}

% Commentary command
\newcommand{\commentary}[1]{
  \par\noindent
  {\fontsize{8}{10}\selectfont\textcolor{red}{#1}}
  \par
}

% Red text command
\newcommand{\redtext}[1]{
  \par\noindent
  \textcolor{red}{#1}
  \par
}

% Red dash command
\newcommand{\reddash}{\textcolor{red}{\textemdash{}}\xspace}
\usepackage{xspace}

% Lectio title command
\newcommand{\lectiotitle}[1]{
  \par\noindent
  {\centering\textcolor{red}{\textit{#1}}\par}
  \setlength{\parindent}{0.2in}
  \everypar{\setlength{\parindent}{0.2in}}
}

% Versicle-Response command
\newcommand{\versicle}[2]{
  \par\noindent
  \hangindent=1em\hangafter=1
  \emergencystretch=1em
  \makebox[1em][l]{\textcolor{red}{V.}}#1
  \par\noindent
  \hangindent=1em\hangafter=1
  \emergencystretch=1em
  \makebox[1em][l]{\textcolor{red}{F.}}#2
  \par
}

% Hymn stanzas environment
\usepackage{multicol}
\newenvironment{hymnstanzas}{
  \setlength{\leftmargin}{0pt}
  \setlength{\rightmargin}{0pt}
  \setlength{\columnsep}{4em}
  \begin{multicols}{2}
  \raggedright
}{
  \end{multicols}
}

\newcommand{\stanzabreak}{%
  \par\addvspace{0.8\baselineskip}%
}


% Single centered stanza (for odd last stanza)
\newcommand{\centerstanza}[1]{
  \par
  \begin{center}
  \begin{minipage}{0.4\textwidth}
  \raggedright
  #1
  \end{minipage}
  \end{center}
  \par
}

% Structured text environment
\newenvironment{preces}[2]{
  \par\noindent
  #1
  \par\noindent
  \hskip0.2in\textit{#2}
  \par
  \setlength{\parindent}{0pt}
  \everypar{\hangindent=0.2in\hangafter=1}
}{
  \par
}

\usepackage{ragged2e}


\title{Tábori zsolozsmáskönyv}
\subtitle{902. Kucsera Ferenc cserkészcsapat - Gróf Széchenyi István raj - 2026 téli tábor}

\begin{document}

\maketitle

\section*{Általános rész}

\subsubsection*{Reggeli dicséret}

\parttitle{Imaóra kezdete}

\versicle{Istenem, jöjj segítség\underline{e}mre!}{%
  Uram, segíts meg \underline{e}ngem! Dicsőség az Atyának a Fiúnak és a Szentlél\underline{e}knek, %
  miképpen kezdetben, most és mindörökké. \underline{A}men.%
}

\subsubsection*{Esti dicséret}

\commentary{Imaóra kezdete mint a Reggeli dicséretben.}

\section*{Zsoltáros rész}

\subsection*{Péntek}

\parttitle{Himnusz}

% Hymnus: Iesu, quadragenariae
\begin{gregosheet}
  \melody{<-3--2--4----tT4---3----4t--tT4R3-2-:-1--1t--5---5---tT4-4z-tT4-4t--;-5--5----4--tT4-3---4----eE2-1--:-3----2--4---tT4-3---4t-tT4R3-2-.}
  \lyrics{  Úr Jé-zus, szent negy-ven-na-pos    böj-töd vál-lal-juk új-ra most: üd-vünk-re pél-dát adsz ne-künk, hogy fé-kez-zük ter-mé-sze-tünk.}
\end{gregosheet}

\begin{hymnstanzas}
  Légy hát az Egyház támasza, \\
  légy bánatunknak orvosa; \\
  megtört szívünk kér gyógyulást: \\
  adj bánat szülte tisztulást!
\stanzabreak
  Áraszd ránk bő kegyelmedet, \\
  bocsáss meg régi bűnöket, \\
  s lelkünket újaktól te óvd, \\
  örök jóságú Alkotónk!
\stanzabreak
  Kérünk, hogy évi böjtjeink \\
  és más vezeklő tetteink \\
  számunkra tőled nyerjenek \\
  boldog húsvéti ünnepet!
\stanzabreak
  Kegyes Háromság, tégedet \\
  imádjanak már mindenek, \\
  s kiket megújít nagy kegyed, \\
  zengjük dicső új éneked. Ámen.
\end{hymnstanzas}

\redtext{1. antifóna}

% Antiphona: Inclinavit Dominus
\begin{gregosheet}
  \melody{<-3---3----4--tT4-5-:-2-3---4--eE2-1---1---,,-5-5-4-3-4t-4.}
  \lyrics{  Meg-hall-ja az  Úr  i-mád-sá-gom sza-vát.}
\end{gregosheet}

\begin{psalmsheet}[tone=1, continuous, number={114. zsoltár}, title={Hálaadás}, motto={Sok szorongatás közepette kell bejutnunk az Isten országába (ApCsel 14, 21).}]
  \psalmtext{
    1. Szeretem az Urat, * mert meghallgatta esdeklő szavamat. \\
    2. Hozzám fordította fülét, * amikor segítségül hívtam a nehéz napokban. ---

    3. A halál kötelékei körülfontak engem, * és elértek az alvilág csapdái. \\
    4. Gyötrelem és fájdalom várt rám mindenütt, † az Úr nevét kiáltottam akkor: * „Uram, mentsd meg az életem!” ---

    5. Igazságos az Úr és jóságos, * irgalmas hozzánk a mi Istenünk. \\
    6. Oltalmába fogadja a gyöngét; * nyomorult voltam, de ő fölemelt. ---

    7. Lelkem, nyugodj meg újra, * mert az Úr jót tett veled. \\
    8. Megmentette életemet a haláltól, † szememet a könnyhullatástól, * és lábamat az elbukástól. \\
    9. Az Úr előtt járhatok * az élők földjén.
}
\end{psalmsheet}

\redtext{2. antifóna}

% Antiphona: Auxilium meum
\begin{gregosheet}
  \melody{<-3--eE2W1-1-:-0-1--3--3---0e-1----,,-3--3--0q--1.}
  \lyrics{  Az Úr-tól    a mi se-gít-sé-günk.}
\end{gregosheet}

\begin{psalmsheet}[tone=2, continuous, number={120. zsoltár}, title={Népünk őrzője}, motto={Nem éheznek, és nem szomjaznak többé, a nap nem égeti őket, sem másfajta hőség (Jel 7, 16).}]
  \psalmtext{
    1. Tekintetem a hegyek felé emelem, * honnét is jöhetne segítségem? \\
    2. Segítséget csak az Úrtól kaphatok, * aki az eget és a földet teremtette. ---

    3. Ő nem hagyja, hogy meginogjon lábad, * nem szunnyad az, aki őriz téged. \\
    4. Bizony sosem szunnyad, sosem alszik, * aki őre Izraelnek. ---

    5. Az Úr a te őrződ, † az Úr oltalmaz téged, * ő áll jobbod felől. \\
    6. Napközben a nap heve nem éget, * és a hold nem árt neked éjjel. ---

    7. Megvéd az Úr minden bajtól, * és megőrzi lelkedet. \\
    8. Az Úr megőriz jártodban-keltedben, * mostantól fogva és mindörökké.
}
\end{psalmsheet}

\redtext{3. antifóna}

% Antiphona: In consilio iustorum
\begin{gregosheet}
  \melody{<-4--4-4--4---5--4--5---4-:-3t-7---7--6--5--4u-zZ5-:-4--4----5--5--4--4--4-,,-7-7-6-7-5-4-.}
  \lyrics{  Az i-ga-zak ta-ná-csá-ban és gyü-le-ke-ze-té-ben   na-gyok az Úr mű-ve-i.}
\end{gregosheet}

\begin{psalmsheet}[tone=8, continuous, title={A hódolat éneke}]
  \psalmtext{
    3. Nagy és csodálatos minden műved, * mindenható Urunk, Istenünk. \\
    3. Hűségesek és igazak útjaid, * nemzetek Királya! --- % sic!

    4. Ki ne félne, tisztelne téged, Urunk, * ki ne dicsérné nevedet? \\
    4. Egyedül csak te vagy a Szent, † eljön minden nemzet, és színed elé borul, * mert nyilvánvalóvá lett, ahogyan ítéltél.
}
\end{psalmsheet}

\end{document}
