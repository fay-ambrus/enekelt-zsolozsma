\documentclass{article}
\usepackage{gregosheet}
\usepackage{gregosheet-psalm}
% Font sizes
\newcommand{\musicfontsize}{20}
\newcommand{\lyricfontsize}{10}
\newcommand{\psalmfontsize}{10}

% Fonts
% Change this to any font command you want:
% \rmfamily (roman/serif), \sffamily (sans-serif), \ttfamily (monospace)
% or use \setmainfont{FontName} or \newfontfamily for specific fonts
\newfontfamily\lyricfont{Cambria}
\newfontfamily\psalmfont[BoldFont={Cambria Bold}]{Cambria}
\newfontfamily\headingfont[BoldFont={Cambria Bold}]{Cambria}
\setmainfont{Cambria}

% Spacing
\newcommand{\systemvskip}{10pt}
\newcommand{\lyricvskip}{2pt}
\newcommand{\blockvskip}{10pt}

% Page layout
\usepackage[a5paper,margin=0.5in]{geometry}
\pagestyle{empty}
\setlength{\parindent}{0pt}

% Title style
\makeatletter
\renewcommand{\maketitle}{
  \begin{center}
    \headingfont\fontsize{24}{28}\selectfont\bfseries\MakeUppercase{\@title}
    \vskip 1em
    \headingfont\fontsize{13}{16}\selectfont\MakeUppercase{\@subtitle}
  \end{center}
}
\newcommand{\subtitle}[1]{\gdef\@subtitle{#1}}
\newcommand{\@subtitle}{}
\makeatother

% Section style
\usepackage{titlesec}
\titleformat{\section}
  {\centering\headingfont\fontsize{20}{24}\selectfont\bfseries}
  {}{0pt}{\MakeUppercase}
\titlespacing*{\section}{0pt}{\blockvskip}{\blockvskip}

% Subsection style
\titleformat{\subsection}
  {\centering\headingfont\fontsize{16}{20}\selectfont\bfseries\addfontfeature{LetterSpace=10.0}}
  {}{0pt}{\MakeUppercase}
\titlespacing*{\subsection}{0pt}{\blockvskip}{\blockvskip}

% Subsubsection style
\titleformat{\subsubsection}
  {\centering\headingfont\fontsize{12}{14}\selectfont\bfseries\scshape}
  {}{0pt}{}
\titlespacing*{\subsubsection}{0pt}{\blockvskip}{\blockvskip}

% Part title command
\newcommand{\parttitle}[2][]{
  \vskip\blockvskip
  \par\noindent
  \headingfont\fontsize{10}{12}\selectfont
  \textcolor{red}{\MakeUppercase{#2}}%
  \ifx&#1&\else\hfill\textcolor{red}{#1}\fi
  \par
}

% Commentary command
\newcommand{\commentary}[1]{
  \par\noindent
  {\fontsize{8}{10}\selectfont\textcolor{red}{#1}}
  \par
}

% Red text command
\newcommand{\redtext}[1]{
  \par\noindent
  \textcolor{red}{#1}
  \par
}

% Red dash command
\newcommand{\reddash}{\textcolor{red}{\textemdash{}}\xspace}
\usepackage{xspace}

% Lectio title command
\newcommand{\lectiotitle}[1]{
  \par\noindent
  {\centering\textcolor{red}{\textit{#1}}\par}
  \setlength{\parindent}{0.2in}
  \everypar{\setlength{\parindent}{0.2in}}
}

% Versicle-Response command
\newcommand{\versicle}[2]{
  \par\noindent
  \hangindent=1em\hangafter=1
  \emergencystretch=1em
  \makebox[1em][l]{\textcolor{red}{V.}}#1
  \par\noindent
  \hangindent=1em\hangafter=1
  \emergencystretch=1em
  \makebox[1em][l]{\textcolor{red}{F.}}#2
  \par
}

% Hymn stanzas environment
\usepackage{multicol}
\newenvironment{hymnstanzas}{
  \setlength{\leftmargin}{0pt}
  \setlength{\rightmargin}{0pt}
  \setlength{\columnsep}{4em}
  \begin{multicols}{2}
  \raggedright
}{
  \end{multicols}
}

\newcommand{\stanzabreak}{%
  \par\addvspace{0.8\baselineskip}%
}


% Single centered stanza (for odd last stanza)
\newcommand{\centerstanza}[1]{
  \par
  \begin{center}
  \begin{minipage}{0.4\textwidth}
  \raggedright
  #1
  \end{minipage}
  \end{center}
  \par
}

% Structured text environment
\newenvironment{preces}[2]{
  \par\noindent
  #1
  \par\noindent
  \hskip0.2in\textit{#2}
  \par
  \setlength{\parindent}{0pt}
  \everypar{\hangindent=0.2in\hangafter=1}
}{
  \par
}

\usepackage{ragged2e}

\makeatletter

\newcommand{\printsheet}[1]{%
  \ifcsname sheet@#1\endcsname
    \csname sheet@#1\endcsname
  \else
    \textbf{[Missing sheet: #1]}%
  \fi
}

% ============================================================================
% I. PROPRIUM DE TEMPORE
% ============================================================================

% ----------------------------------------------------------------------------
% III. Tempus Quadragaeismae
% ----------------------------------------------------------------------------

% ----------------------------------------------------------------------------
% a. Usque ad sabbatum hebdomadae quintae
% ----------------------------------------------------------------------------

% --- Pro cunctis diebus ---

% ··· Vesperas ···

% In officio dominicali
% TODO

% In officio feriali

% Hymnus
\newcommand{\sheet@IesuQuadragenariae}{%
  \begin{gregosheet}
    \melody{<-3--2--4----tT4---3----4t--tT4R3-2-:-1--1t--5---5---tT4-4z-tT4-4t--;-5--5----4--tT4-3---4----eE2-1--:-3----2--4---tT4-3---4t-tT4R3-2-.}
    \lyrics{  Úr Jé-zus, szent negy-ven-na-pos    böj-töd vál-lal-juk új-ra most: üd-vünk-re pél-dát adsz ne-künk, hogy fé-kez-zük ter-mé-sze-tünk.}
  \end{gregosheet}
}

% Responsorium breve
\newcommand{\sheet@EgoDixi}{%
  \begin{gregosheet}
    \melody{<X-3---4----3----4t-5---,-4--5---rR3--4---3-,,-5---5-----5---4--4--,-4----4---4---5---rR3-4t-5-,-5--5---5---5--5-5z--5-:-5-5--4-4-:-5--5-4-----3--4t--tT4.}
    \lyrics{   Így szól-tam: U-ram, * Kö-nyö-rülj raj-tam! Gyó-gyíts meg en-gem, mert vét-kez-tem el-le-ned. Di-cső-ség az A-tyá-nak a Fi-ú-nak és a Szent-lé-lek-nek.}
  \end{gregosheet}
}

% ············································································
% Hebdomada II Quadragesimae
% ············································································

% --- Feria VI ---

% ··· Vesperas ···

% Magnificat

\newcommand{\sheet@QuaerentesEumTenere}{%
  \begin{gregosheet}[tone=3. tónus]
    \melody{<-4--5---6--7-5---7---6----:-4--4---5---5-zZ5-4--,-6----6---5--uU6-4-:-5--rR3-rR3-2-,,-7-7-5u-5-4-.}
    \lyrics{  El-fog-ni i-gye-kez-tek, * de fél-tek a nép-től, mert pró-fé-tá-nak tar-tot-ták őt.}
  \end{gregosheet}%
}

% ============================================================================
% II. PSALTERIUM PER QUATTOR HEBDOMADAS DISTRIBUTUM
% ============================================================================

% ············································································
% Hebdomada II
% ············································································

% --- Feria VI ---

% ··· Vesperas ···

% Hymnus
% TODO

% Antiphona
\newcommand{\sheet@InclinavitDominus}{%
  \begin{gregosheet}[tone=1. ré tónus]
    \melody{<-3---3----4--tT4-5-:-2-3---4--eE2-1---1---,,-5-5-4-3-4t-4.}
    \lyrics{  Meg-hall-ja az  Úr  i-mád-sá-gom sza-vát.}
  \end{gregosheet}%
}

% Antiphona
\newcommand{\sheet@AuxiliumMeum}{%
  \begin{gregosheet}[tone=2. tónus]
    \melody{<-3--eE2W1-1-:-0-1--3--3---0e-1----,,-3--3--0q--1.}
    \lyrics{  Az Úr-tól    a mi se-gít-sé-günk.}
  \end{gregosheet}%
}

% vel ad libitum:
% TODO

% Antiphona
\newcommand{\sheet@InConsilioIustorum}{%
  \begin{gregosheet}[tone=8. szó tónus]
    \melody{<-4--4-4--4---5--4--5---4-:-3t-7---7--6--5--4u-zZ5-:-4--4----5--5--4--4--4-,,-7-7-6-7-5-4-.}
    \lyrics{  Az i-ga-zak ta-ná-csá-ban és gyü-le-ke-ze-té-ben   na-gyok az Úr mű-ve-i.}
  \end{gregosheet}%
}

\makeatother

\title{Tábori zsolozsmáskönyv}
\subtitle{902. Kucsera Ferenc cserkészcsapat - Gróf Széchenyi István raj - 2026. téli tábor}

\begin{document}

\maketitle

\section*{Általános rész}

\subsubsection*{Imádságra hívás}

\textcolor{red}{(}Nyisd meg Uram, ajkamat neved dicséretére; tisztítsd meg szívemet minden
hiába\-való és vétkes gondolattól; világosítsd meg értelmemet, gyújtsd lángra szívemet, hogy ezt a
zsolozsmát méltón, figyelmesen, áhítattal végezzem, és meghallgatásra méltó legyek isteni fölséged
színe előtt. Krisztus, a mi Urunk által.\textcolor{red}{)}

\parttitle{Imaóra kezdete}

\versicle{Nyisd meg, Uram, ajkam\underline{a}t,}{Hogy dicséretedet hirdesse szav\underline{a}m.}

\printsheet{ChristumDominumProNobis}

\subsubsection*{Olvasmányos imaóra}

\parttitle{Imaóra kezdete}

\commentary{%
  Imaóra kezdete mint a Reggeli dicséretben, azonban elmarad, ha az Imádságra hívás az Olvasmányos %
  imaórát előzi meg.%
}

\subsubsection*{Reggeli dicséret}

\parttitle{Imaóra kezdete}

\versicle{Istenem, jöjj segítség\underline{e}mre!}{%
  Uram, segíts meg \underline{e}ngem! Dicsőség az Atyának a Fiúnak és a Szentlél\underline{e}knek, %
  miképpen kezdetben, most és mindörökké. \underline{A}men.%
}

\parttitle{Imaóra befejezése}

\versicle{%
  Az Úr áldjon meg minket, védelmezzen minden rossztól és vezessen el az örök életr\underline{e}!%
}{%
  Am\underline{e}n.%
}

\subsubsection*{Esti dicséret}

\commentary{Imaóra kezdete mint a Reggeli dicséretben.}

\commentary{Imaóra befejezése mint a Reggeli dicséretben.}

\subsubsection*{Befejező imaóra}

\commentary{Imaóra kezdete mint a Reggeli dicséretben.}

\parttitle{Lelkiismeret-vizsgálat}

\commentary{%
  Dicséretes dolog most a lelkiismeret-vizsgálat, amely közös végzés esetén a mise bűnbánati %
  formája szerint történhet.%
}

\parttitle{Himnusz}

\commentary{Szombaton és vasárnap este:}

\printsheet{TeLucisAnteTerminum}

\begin{hymnstanzas}
  Álmodjék rólad a szívünk, \\
  álmunkban is te légy velünk, \\
  téged dicsérjen énekünk, \\
  midőn új napra ébredünk!
\stanzabreak
  Adj nekünk üdvös életet, \\
  szítsd fel a szív fáradt tüzét, \\
  a ránk törő rút éjhomályt \\
  világosságod rontsa szét!
\end{hymnstanzas}
\centerstanza{%
  Kérünk, mindenható Atyánk, \\
  Úr Jézus Krisztus érdemén, \\
  ki Szentlélekkel és veled \\
  uralkodik, s örökre él. Ámen.
}

\parttitle{Rövid válaszos ének}

\printsheet{InManusTuasDomine}

\parttitle[Lk 2,29-32]{Nunc dimittis}

\printsheet{SalvaNos}

\commentary{Minden sor intiummal.}

\begin{psalmsheet}[tone=1, continuous, initium, title={Krisztus a pogányok világossága és Izrael dicsősége}]
  \psalmtext{
    29. Most bocsátod el szolgádat, Uram, * a te igéd szerint békességben. \\
    30. (!) Mert látták szemeim * a te üdvösségedet. \\
    31. (!) Akit rendeltél * minden népek színe előtt, \\
    32. Hogy fény legyen a népek világosságára, * és dicsőségére népednek, Izráelnek. \\
    33. Dicsőség az Atyának és Fiúnak * és Szentlélek Istennek, \\
    34. Miképpen kezdetben vala, most és mindenkor * és mindörökkön örökké! Ámen.
}
\end{psalmsheet}

\parttitle{Imaóra befejezése}

\versicle{%
 A nyugodalmas éjszakát és a jó halál kegyelmét adja meg nekünk a mindenható Ist\underline{e}n!%
}{Am\underline{e}n.}

\parttitle{Szűz Mária záróantifónája}

\printsheet{AveReginaCaelorum}

\section*{Zsoltáros rész}

\subsection*{Péntek}

\subsubsection*{Esti dicséret}

\parttitle{Himnusz}

\printsheet{IesuQuadragenariae}

\begin{hymnstanzas}
  Légy hát az Egyház támasza, \\
  légy bánatunknak orvosa; \\
  megtört szívünk kér gyógyulást: \\
  adj bánat szülte tisztulást!
\stanzabreak
  Áraszd ránk bő kegyelmedet, \\
  bocsáss meg régi bűnöket, \\
  s lelkünket újaktól te óvd, \\
  örök jóságú Alkotónk!
\stanzabreak
  Kérünk, hogy évi böjtjeink \\
  és más vezeklő tetteink \\
  számunkra tőled nyerjenek \\
  boldog húsvéti ünnepet!
\stanzabreak
  Kegyes Háromság, tégedet \\
  imádjanak már mindenek, \\
  s kiket megújít nagy kegyed, \\
  zengjük dicső új éneked. Ámen.
\end{hymnstanzas}

\redtext{1. antifóna}

\printsheet{InclinavitDominus}

\begin{psalmsheet}[tone=1, continuous, number={114. zsoltár}, title={Hálaadás}, motto={Sok szorongatás közepette kell bejutnunk az Isten országába (ApCsel 14, 21).}]
  \psalmtext{
    1. Szeretem az Urat, * mert meghallgatta esdeklő szavamat. \\
    2. Hozzám fordította fülét, * amikor segítségül hívtam a nehéz napokban. ---

    3. A halál kötelékei körülfontak engem, * és elértek az alvilág csapdái. \\
    4. Gyötrelem és fájdalom várt rám mindenütt, † az Úr nevét kiáltottam akkor: * „Uram, mentsd meg az életem!” ---

    5. Igazságos az Úr és jóságos, * irgalmas hozzánk a mi Istenünk. \\
    6. Oltalmába fogadja a gyöngét; * nyomorult voltam, de ő fölemelt. ---

    7. Lelkem, nyugodj meg újra, * mert az Úr jót tett veled. \\
    8. Megmentette életemet a haláltól, † szememet a könnyhullatástól, * és lábamat az elbukástól. \\
    9. Az Úr előtt járhatok * az élők földjén.
}
\end{psalmsheet}

\redtext{2. antifóna}

\printsheet{AuxiliumMeum}

\begin{psalmsheet}[tone=2, continuous, number={120. zsoltár}, title={Népünk őrzője}, motto={Nem éheznek, és nem szomjaznak többé, a nap nem égeti őket, sem másfajta hőség (Jel 7, 16).}]
  \psalmtext{
    1. Tekintetem a hegyek felé emelem, * honnét is jöhetne segítségem? \\
    2. Segítséget csak az Úrtól kaphatok, * aki az eget és a földet teremtette. ---

    3. Ő nem hagyja, hogy meginogjon lábad, * nem szunnyad az, aki őriz téged. \\
    4. Bizony sosem szunnyad, sosem alszik, * aki őre Izraelnek. ---

    5. Az Úr a te őrződ, † az Úr oltalmaz téged, * ő áll jobbod felől. \\
    6. Napközben a nap heve nem éget, * és a hold nem árt neked éjjel. ---

    7. Megvéd az Úr minden bajtól, * és megőrzi lelkedet. \\
    8. Az Úr megőriz jártodban-keltedben, * mostantól fogva és mindörökké.
}
\end{psalmsheet}

\redtext{3. antifóna}

\printsheet{InConsilioIustorum}

\begin{psalmsheet}[tone=8, continuous, title={A hódolat éneke}]
  \psalmtext{
    3. Nagy és csodálatos minden műved, * mindenható Urunk, Istenünk. \\
    3. Hűségesek és igazak útjaid, * nemzetek Királya! --- % sic!

    4. Ki ne félne, tisztelne téged, Urunk, * ki ne dicsérné nevedet? \\
    4. Egyedül csak te vagy a Szent, † eljön minden nemzet, és színed elé borul, * mert nyilvánvalóvá lett, ahogyan ítéltél.
}
\end{psalmsheet}

\parttitle[Jak 5,16.19-20]{Rövid olvasmány}

Valljátok meg egymásnak bűneiteket, és imádkozzatok egymásért, hogy meggyó\-gyulj\underline{a}t\underline{o}k. † Igen hathatós az igaz ember buzgó k\underline{ö}nyörgése. * Testvéreim, ha valaki közületek letér az igazság útjáról, és akad, aki visszatéríti, az tudja meg, hogy aki a bűnöst letéríti a tévelygés útjáról, megmenti lelkét a haláltól, és tömérdek bűnre fáty\underline{o}lt b\underline{o}rít.

\parttitle{Rövid válaszos ének}

\printsheet{EgoDixi}

\parttitle[Lk 1,46-55]{Magnificat}

\printsheet{QuaerentesEumTenere}

\commentary{Minden sor intiummal.}

\begin{psalmsheet}[tone=3, continuous, initium, title={A lélek az Úrban ujjong}]
  \psalmtext{
    1. (+) Magasztalja lelkem az Urat, * és szívem ujjong üdvözítő Istenemben. \\
    2. Tekintetre méltatta alázatos szolgálóleányát: * Íme, ezentúl boldognak hirdetnek az összes nemzedékek, \\
    3. mert nagyot művelt velem ő, aki Hatalmas: * ő, akit Szentnek hívunk. \\
    4. Nemzedékről nemzedékre megmarad irgalma azokon, * akik istenfélők. ---

    5. Csodát művelt erős karjával: * a kevélykedőket széjjelszórta, \\
    6. hatalmasokat elűzött trónjukról, * kicsinyeket pedig felmagasztalt; \\
    7. az éhezőket minden jóval betölti, * a gazdagokat elbocsátja üres kézzel. ---

    8. Gondjába vette gyermekét, Izraelt: * megemlékezett irgalmáról, \\
    9. melyet atyáinknak hajdan megígért, * Ábrahámnak és utódainak mindörökké.
}
\end{psalmsheet}

\parttitle{Fohászok}

\begin{preces}{%
  Imádjuk az emberi nem Megváltóját, aki halálával legyőzte a halált, és feltámadásával %
  újjáalkotta az életet. Kérjük alázatos lélekkel.%
}{Szenteld meg népedet, amelyet véreddel megváltottál!}
  Isteni Megváltónk, engedd, hogy a bűnbánattal szorosabban kapcsolódjunk szenve\-désedhez,

  \reddash és így eljussunk a feltámadás dicsőségére!

  Engedd, hogy elnyerjük Édesanyádnak, a szomorúak vigasztalójának pártfogását,

  \reddash és erősítsük a szomorkodókat azzal a vigasztalással, amellyel te erősítesz minket!

  Add meg híveidnek, hogy megpróbáltatásaik által szenvedésed részeseivé legyenek,

  \reddash s így valósuljon meg bennük megváltásod!

  Te megaláztad magad, és engedelmes lettél a halálig, mégpedig a kereszthalálig,

  \reddash növeld szolgáidban az engedelmességet és a türelmet!

  Elhunyt testvéreinket részesítsd tested dicsőséges feltámadásában,

  \reddash és egykor tégy minket is társaikká!
\end{preces}
Mi Atyánk...

\parttitle{Könyörgés}

Mindenható Istenünk, adj erőt, hogy a közelgő ünnepekre töredelmes bűnbánattal tisztítsuk meg
lelkünket. A mi Urunk, Jézus Krisztus, a te Fiad által, aki veled él és uralkodik a Szentlélekkel
egységben, Isten mindörökkön-örökké.


\subsubsection*{Befejező imaóra}

\parttitle{Himnusz}

\commentary{Az általános részben található himnusz dallamára.}

\begin{hymnstanzas}
  Krisztus, tündöklő nappalunk, \\
  az éj árnyát eloszlatod, \\
  kit fényből fénynek vall a hit \\
  és fénnyel áldod szentjeid.
\stanzabreak
  Esengünk hozzád, szent Urunk, \\
  míg tart az éj, vigyázz reánk, \\
  találjunk benned pihenést, \\
  adj nyugodalmas éjszakát.
\stanzabreak
  Mikor lecsukjuk két szemünk, \\
  a szívünk virrasszon veled, \\
  őrködjék mindig jobb kezed \\
  hűséges híveid felett.
\stanzabreak
  Védelmezőnk, tekints le ránk, \\
  igázd le ellenségeink, \\
  vezesd szolgáidat, kikért \\
  véred váltsága volt a bér.
\end{hymnstanzas}
\centerstanza{%
  Legyen most néked, Krisztusunk, \\
  és szent Atyádnak tisztelet, \\
  s a Szentléleknek is veled \\
  zengjen dicséret szüntelen. Ámen.
}

\parttitle{Zsoltározás}

\redtext{Antifóna}

\printsheet{DomineDeusSalutisMeae}

\begin{psalmsheet}[tone=4, continuous, number={87. zsoltár }, title={Súlyos beteg imája}, motto={Ez a ti órátok, a sötétség hatalmáé (Lk 22, 53).}]
  \psalmtext{
    2. Uram, szabadító Istenem, * éjjel-nappal tehozzád kiáltok. \\
    3. Imádságom jusson színed elé, * hajlítsd füledet kérésemre! ---

    4. Mert a lelkem tele gyötrelemmel, * életem közel van a holtak országához. \\
    5. A sírba szállók közé számítanak, * olyan ember vagyok, aki a végét járja. \\
    6. A halottak között van a fekhelyem, * mint aki megsebesítve a sírban alszik, \\
    7. akinek már emlékét sem őrzik többé, * és aki már kiszakadt a kezedből. ---

    8. A verem mélyére vetettél, * a sötétségbe és a halál árnyékába. \\
    9. Rám nehezedett súlyos haragod, * örvényeid fölöttem összecsapnak. ---

    10. Ismerőseimet távol tartod tőlem, * utálatossá tettél előttük. \\
    11. Fogoly vagyok, szabadulásom nincsen, * szemem elsorvad a kíntól. \\
    12. Uram, egész nap hozzád kiáltok, * és kezemet feléd tárom. ---

    13. Csodákat talán a holtaknak teszel, * vagy az árnyak fölkelnek, hogy áldjanak? \\
    14. Vajon irgalmadat hirdeti valaki a sírban, * vagy hűségedet a pusztulás helyén? \\
    15. Vajon csodáidat meglátja valaki a sötétben, * vagy igazságodat a feledés földjén? ---

    16. De én hozzád kiáltok, Uram, * már kora reggel eléd siet imádságom. \\
    17. Miért taszítasz el magadtól, Uram, * arcodat miért rejted el tőlem? \\
    18. Nyomorult vagyok, és halálra vált ifjúkorom óta, * megzavart és földre nyomott csapásaid súlya. \\
    19. Rám zúdult haragod, * és feldúlt a tőled való rettegés. \\
    20. Hullámzik körülöttem egész nap, mint az árvíz, * már-már összecsap fölöttem. \\
    21. Elszakítottad tőlem jó barátaimat, * már csak a sötétség ismer engem.
}
\end{psalmsheet}

\parttitle[Jer 14, 9]{Rövid olvasmány}
Itt élsz közöttünk, Ur\underline{a}m, † a te nev\underline{e}det viseljük. * Ne hagyj el hát minket, Urunk, \underline{I}st\underline{e}nünk!

\parttitle{Könyörgés}
Mindenható Istenünk, add, hogy sírjában nyugvó egyszülött Fiad mellett hűségesen kitartsunk, és
méltók legyünk vele együtt új életre támadni. Aki él és uralkodik mindörökkön-örökké.

\subsection*{Szombat}

\subsubsection*{Olvasmányos imaóra}

\parttitle{Himnusz}

\printsheet{NuncTempusAcceptabile}

\parttitle{Zsoltározás}

\redtext{1. antifóna}

\printsheet{VisitaNosDomine}

\begin{psalmsheet}[tone=8, continuous, number={105. zsoltár }, title={Az Úr jósága, a nép hűtlensége}, motto={Mindez előkép számunkra, a mi okulásunkra írták meg, akik a végső időkben élünk (1 Kor 10, 11).}, numeral={I.}]
  \psalmtext{
    1. Adjatok hálát az Úrnak, mert jó, * irgalma örökké megmarad. ---

    2. Ki tudná elmondani az Úr hatalmas tetteit, * dicséretét hiánytalanul hirdetni ki képes? \\
    3. Boldog, aki megtartja a törvényt, * akinek tettei mindenkor igazak. ---

    4. Gondolj ránk, Urunk, hiszen szereted népedet, * látogass hozzánk segítségeddel; \\
    5. hadd lássuk választottaid boldogságát, † hadd örvendjünk nemzeted örömének, * és dicsekedjünk örökségeddel. ---

    6. Atyáinkkal együtt mi is vétkeztünk, * bűnbe estünk, gonoszat cselekedtünk. \\
    7. Atyáink Egyiptomban nem értették meg csodáidat, † mérhetetlen irgalmadra nem gondoltak, * és lázadoztak, amikor elérték a Vörös-tengert. \\
    8. De az Úr megszabadította őket nevéért, * hogy megismertesse hatalmát. ---

    9. Rákiáltott a Vörös-tengerre, és az kiszáradt, * a mély tengeren, akár a pusztán, vezette át őket. \\
    10. Gyűlölőik hatalmából kimentette őket, * s megszabadította ellenségeik kezéből. \\
    11. Sanyargatóikat elnyelte a tenger, * egy sem maradt életben közülük. \\
    12. Hittek ezért ígéreteiben, * és dicséretét énekelték. ---

    13. De csakhamar megfeledkeztek tetteiről, * és nem bíztak Isten terveiben. \\
    14. Telhetetlen vágyra gerjedtek a pusztában, * a sivatagban Istent kísértették. \\
    15. Kielégítette akkor sóvárgásukat, * és torkig jóllakatta őket. ---

    16. A táborban pártot ütöttek Mózes ellen, * és Áron ellen, aki az Úr szentje. \\
    17. De meghasadt a föld, s elnyelte Datánt, * és eltemette Abiron hadát. \\
    18. Lobogó láng csapott fel seregük ellen, * láng emésztette el a bűnösöket.
}
\end{psalmsheet}

\redtext{2. antifóna}

\printsheet{OblitiSuntDeum}

\begin{psalmsheet}[tone=1, continuous, numeral={II.}]
  \psalmtext{
    19. Borjút öntöttek Hóreb hegyénél, * és leborultak az arany képmás előtt. \\
    20. Dicsőségüket fölcserélték * a füvet legelő borjú képével. \\
    21. Elfeledték szabadító Istenüket, * aki Egyiptomban jeleket művelt, \\
    22. Kám földjén csodákat, * a Vörös-tengernél félelmetes dolgokat. \\
    23. Már arra gondolt, hogy eltörli őket, * ha nem lett volna Mózes, akit kiválasztott. \\
    23. Ám ő résen állt színe előtt, * elfordította az Úr haragját, hogy ne ártson nekik. ---

    24. Semmire sem becsülték a megígért földet, * nem hittek ígéreteinek. \\
    25. Zúgolódva ültek sátraik mélyén, * nem törődtek Isten szavával. \\
    26. Ekkor fölemelt kézzel megesküdött nekik, * hogy eltiporja őket a pusztában, \\
    27. szétszórja nemzetségüket a pogányok között, * elszéleszti őket messze országokba. ---

    28. Aztán meg Baalpeort kezdték imádni, * holt bálványnak adott áldozatból lakomáztak. \\
    29. Minden tettükkel Istent ingerelték, * romlás zúdult ezért fejükre. \\
    30. Fölállt akkor Pinehász, és ítéletet tartott, * mire megszűnt a csapás. \\
    31. Ez érdemül számított neki * nemzedékeken át, mindörökre. ---

    32. Meriba vizénél is felháborították, * még Mózest is bajba sodorták, \\
    33. mert keserűségbe hajszolták lelkét, * és ő meggondolatlan szóra nyitotta száját.
}
\end{psalmsheet}

\redtext{3. antifóna}

\printsheet{SalvosNosFacDomine}

\begin{psalmsheet}[tone=1, continuous, numeral={III.}]
  \psalmtext{
    34. Nem irtották ki a pogányokat sem, * noha az Úr meghagyta nekik. \\
    35. Elvegyültek a pogányok közé, * és cselekedeteiket eltanulták. \\
    36. Bálványaik szolgálatára adták magukat, * és ez lett végül a vesztük. ---

    37. Saját fiaikat áldozatul adták, * és lányaikat a démonoknak. \\
    38. Ártatlan vért ontottak, † saját fiaik és lányaik vérét; * Kánaán bálványainak áldozták őket. \\
    39. Vérrel szennyezték be az országot, † bemocskolták magukat cselekedeteikkel, * hitszegőkké váltak tetteik által. ---

    40. Haragra lobbant akkor népe ellen az Úr, * és megutálta örökségét. \\
    41. Pogányok kezére adta őket, * gyűlölőik uralkodtak rajtuk. \\
    42. Ellenségeik sanyargatták őket, * megalázást szenvedtek azok keze alatt. ---

    43. Újra meg újra mentésükre jött, † ők azonban szándékaikkal tovább ingerelték, * és bűneikbe visszaestek. \\
    44. Ám ismét rátekintett nyomorúságukra, * és meghallgatta könyörgésüket. ---

    45. Visszaemlékezett a velük kötött szövetségre, * nagy szeretettel megkönyörült rajtuk. \\
    46. Szánalmat ébresztett azok szívében, * akik fogságba hurcolták őket. \\
    47. Szabadíts meg minket, Urunk, Istenünk, * és gyűjts egybe a pogányok közül, \\
    47. hogy szent nevedet magasztaljuk, * és dicséreted legyen dicsekvésünk. ---

    48. Áldott legyen az Úr, Izrael Istene, † öröktől fogva mindörökké. * Az egész nép mondja: „Úgy legyen! Úgy legyen!”
}
\end{psalmsheet}

\versicle{Aki az igazsághoz szabja tetteit, a világosságra m\underline{e}gy,}
  {Hadd derüljön fény a tetteir\underline{e}.}

\parttitle[20,1-17]{Első olvasmány}
A Kivonulás könyvéből

\begin{lectio}{Isten kihirdeti a törvényt a Sínai-hegyen}
\noindent A Sínai-hegyen az Úr ezeket jelentette ki: „Én vagyok az Úr, a te Istened, én hoztalak %
ki Egyiptom földjéről, a szolgaság házából.

Senki mást ne tekints Istennek, csak engem!

Ne csinálj magadnak faragott képet vagy hasonmást arról, ami fent van az égben, vagy lent a %
földön, vagy a vizekben a föld alatt! Ne borulj le ilyen képek előtt, és ne tiszteld őket, mert én, az Úr, a te Istened, féltékeny Isten vagyok. Azoknak vétkét, akik gyűlölnek engem, megtorlom fiaikon, unokáikon és dédunokáikon. De ezredízig irgalmasságot gyakorlok azokkal, akik szeretnek, és megtartják parancsaimat.

Uradnak, Istenednek a nevét ne vedd hiába, mert az Úr nem hagyja büntetlenül azt, aki nevét hiába %
veszi.

Gondolj a szombatra, és szenteld meg! Hat napig dolgozzál, és végezd minden munkádat! A hetedik %
nap azonban az Úrnak, a te Istenednek a pihenő napja, ezért semmiféle munkát nem szabad végezned, sem neked, sem fiadnak, sem lányodnak, sem szolgádnak, sem szolgálólányodnak, sem állatodnak, sem a kapuidon belül tartózkodó idegennek. Az Úr ugyanis hat nap alatt teremtette az eget és a földet, a tengert és mindent, ami bennük van; a hetedik napon azonban megpihent. Az Úr a hetedik napot megáldotta és megszentelte.

Tiszteld apádat és anyádat, hogy sokáig élj azon a földön, amelyet az Úr, a te Istened ad neked.

Ne ölj!

Ne törj házasságot!

Ne lopj!

Ne tégy hamis tanúságot embertársad ellen!

Ne kívánd el embertársad házát, ne kívánd el embertársad feleségét, sem szolgáját, sem %
szolgálólányát, sem szarvasmarháját, sem szamarát, sem más egyebet, ami az övé!”
\end{lectio}

\parttitle[Zsolt 18,8. 9; Róm 13,8. 10]{Válaszos ének}

\versicle{%
  Az Úr törvénye szeplőtelen: megújítj\underline{a} a l\underline{e}lket; † az Úr rendelése %
  hűséges: megokosítja a t\underline{u}d\underline{a}tlant. * Az Úr parancsa világos: fényt %
  gy\underline{ú}jt a sz\underline{e}mnek.%
}{%
  Aki embertársát szer\underline{e}ti, † a többi törvényt is megt\underline{a}rtja. * A törvény %
  tökéletes teljesítése a sz\underline{e}r\underline{e}tet.%
}

\parttitle{Második olvasmány}
Szent Ambrus püspöknek „Elfordulás a világtól” című értekezéséből

{\centering\redtext{(Cap. 6, 36; 7, 44; 8, 45; 9, 52: CSEL 32, 192. 198-199. 204)}}

\begin{lectio}{Ragaszkodjunk Istenhez, az egyetlen igazi jóhoz!}
\noindent Ahol az ember szíve van, ott van a kincse is; az Úr pedig nem szokta megtagadni jó %
adományát azoktól, akik azt kérik.

Tehát mivel az Úr jó, és leginkább azokhoz jó, akik hűségesen kitartanak mellette, ragaszkodjunk %
hozzá, legyünk vele egész lelkünkkel, egész szívünkkel, minden erőnkkel, hogy fényében legyünk, %
lássuk dicsőségét, és élvezzük a mennyei gyönyörűség kegyelmét. Arra a jóra irányítsuk %
figyelmünket, abban legyünk, abban éljünk, és ahhoz ragaszkodjunk, amely meghalad minden értelmet %
és minden megfontolást, meg örök békét és nyugalmat ad; ez a béke pedig minden értelem és %
elgondolás felett áll.

Ez az a jó, amely mindent betölt, s mi mindnyájan benne élünk, és tőle függünk, nála nagyobb %
nincsen, hiszen isteni. Senki sem jó, csak egyedül Isten, tehát ami jó, az isteni, és ami isteni, %
az jó, ezért mondja a Szentírás: \textit{Ha megnyitod kezedet, mindenek jóval telnek el} (Zsolt %
103, 28). Méltán Isten jósága révén kapunk minden olyan jót, amihez rossz nem keveredik.

Ezt a jót ígéri a Szentírás a hívőknek, amikor ezt mondja: \textit{A föld legjavából esztek} (Iz %
1, 19).

Meghaltunk Krisztussal; Krisztus halálát hordozzuk testünkben, hogy Krisztus élete is %
megnyilvánuljon bennünk. Tehát már nem a saját életünket éljük, hanem Krisztusét, az ártatlanság, %
a tisztaság, az egyszerűség életét, és a minden erénnyel teljes életet. Feltámadtunk Krisztussal, %
hogy benne éljünk, hozzá emelkedjünk fel, hogy az a kígyó, amely megsebezte a sarkunkat, újra ránk %
ne találjon a földön.

Meneküljünk innen! Elmenekülhetsz lélekben, még ha a test visszatart is. Képes vagy rá, hogy itt %
is légy, és az Úrnál is legyél, ha hozzá ragaszkodik a lelked, ha gondolataidban utána jársz, ha a %
hitben, még nem látás szerint követed utait, ha hozzá menekülsz. Hiszen menedék és erő az, akihez %
Dávid így szólt: \textit{Hozzád menekültem, és nem csalódtam} (vö. Zsolt 76, 3).

Tehát mivel menedék az Isten - Isten azonban az égben és az egek fölött van -, el kell hát futnunk %
innen oda, ahol béke van, ahol küszködés nélküli nyugalom van, ahol a nagy nyugalom napját %
ünnepeljük, ahogy Mózes mondja: \textit{A föld szombatja is tápláljon téged} (Lev 25, 6). Istenben %
megnyugodni és az ő szépségében gyönyörködni ugyanis olyan, mint egy örömmel és nyugalommal teljes %
lakoma.

Fussunk, mint a szarvasok, a vizek forrásaihoz; amire Dávid szomjazott, szomjazza azt a mi lelkünk %
is. Ki ez a forrás? Halljad: \textit{Tenálad van az élet forrása} (Zsolt 35, 10). Ennek a %
forrásnak mondja a lelkem: \textit{Mikor mehetek, és jelenhetek meg színed előtt?} (vö. Zsolt 41, %
3) A forrás ugyanis az Isten.
\end{lectio}

\parttitle[Mt 22,37; MTörv 10,12]{Válaszos ének}

\versicle{%
  Szeresd Uradat, Istenedet telj\underline{e}s szív\underline{e}dből, † teljes lelkedből és teljes %
  \underline{e}lm\underline{é}dből. * Ez a legnagyobb, az \underline{e}lső p\underline{a}rancs.
}{%
  M\underline{i}t kíván tőled \underline{a}z Úr? † Csak azt, hogy Uradat, Istenedet féld és %
  sz\underline{e}resd, * az Úrnak, a te Istenednek szíved, lelked mélyéb\underline{ő}l %
  sz\underline{o}lgálj.
}

\subsubsection*{Reggeli dicséret}

\parttitle{Himnusz}

\printsheet{IamChristeSolIustitiae}

\begin{hymnstanzas}
  Adtál üdvünkre szent időt, \\
  adj hát bűnbánó szívet is, \\
  hogy jóságod térítse meg, \\
  kiket sokáig tűrt kegyed.
\stanzabreak
  Hadd vezekeljünk némiképp \\
  a bűnbocsánat díjaként: \\
  hisz bűneink bármily nagyok: \\
  csodás irgalmad még nagyobb.
\stanzabreak
  Jön már a nap, fényes napod, \\
  és minden újra felvirít; \\
  mi is vigadjunk boldogan, \\
  hogy szent kegyelmed ránk ragyog.
\stanzabreak
  A nagy mindenség téged áld, \\
  kegyes Háromság, és imád, \\
  bocsánatodban újulók, \\
  mi is mondjunk új éneket! Ámen.
\end{hymnstanzas}

\parttitle{Zsoltározás}

\redtext{1. antifóna}

\printsheet{QuamMagnificataSunt}

\begin{psalmsheet}[tone=8, continuous, number={91. zsoltár}, title={A teremtő Isten dicsérete}, motto={Dicsőítjük az egyszülött Fiút tettei miatt (Szt. Atanáz).}]
  \psalmtext{
    2. Jó dolog az Urat áldani, * énekkel magasztalni nevedet, Fölséges; \\
    3. már hajnalban hirdetni jóságodat, * és éjszaka zengeni hűségedet, \\
    4. tízhúrú lanton és hárfazenével, * citera mellett énekelve. \\
    5. Mert örömmel töltenek el alkotásaid, * kezed művei láttán örvendezem. ---

    6. Mily nagyszerűek a te műveid, Uram, * gondolataid mily mélységesek! \\
    7. Az oktalan ember föl nem fogja, * és nem érti meg őket az esztelen. \\
    8. Sarjadhatnak a bűnösök, akár a fű, * és virágozhatnak a gonosztevők, \\
    9. sorsuk mégis örökös pusztulás; * te azonban, Uram, fölséges vagy örökké. ---

    10. Mert íme, ellenségeid, Uram, † mert íme, ellenségeid elvesznek, * és elpusztulnak mind a gonosztevők. \\
    11. Erőt adsz nekem, bikáéhoz hasonlót, * és fölkentél színtiszta olajjal. \\
    12. Szemem ellenségeimre lenéz, * s a rám törő gonoszok vesztéről hall a fülem. ---

    13. Virul az igaz, akár a pálma, * úgy növekszik, mint a Libanon cédrusa. \\
    14. Az Úr házában van elültetve, * Istenének csarnokában bont virágot. \\
    15. Gyümölcsöt érlel öreg korában is, * termékeny marad, szépen zöldellő. \\
    16. Hirdeti, hogy igaz az Isten, * ő a menedékem, nincs benne hamisság.
}
\end{psalmsheet}

\redtext{2. antifóna}

\parttitle[MTörv 32,1-12]{Kantikum}

\printsheet{DateMagnitudinem}

\begin{psalmsheet}[tone=8, continuous, title={Isten jótéteményei népével}, motto={Hányszor akartam egybegyűjteni fiaidat, ahogy a tyúk szárnya alá gyűjti csibéit (Mt 23, 37).}]
  \psalmtext{
    1. Hallgassátok, egek, szavamat, * hallja meg a föld is szám igéit! \\
    2. Tanításom esőként permetezzen, * szavam úgy szálljon le, mint a harmat, \\
    3. mint mikor zápor hull a rétre, * és a sarjadó füvet eső öntözi. \\
    3. Az Úr nevét hívom segítségül, * magasztaljátok Istenünket! ---

    4. Mert tökéletes minden műve, * és igazságos minden útja. \\
    4. Hűséges és feddhetetlen az Isten, * ő maga az igazság és egyenesség! ---

    5. Vétkeztek ellene, elromlottak, nem fiai már; * gonosz és elfajzott nemzedék ez. \\
    6. Ezzel fizetsz most az Úrnak, * te oktalan, balga nép? \\
    6. Vajon nem tulajdon Atyád ő neked, * aki létrehívott, alkotott, és fönntart? ---

    7. Emlékezz csak a régi időkre, * vedd sorra az egyes nemzedékeket! \\
    7. Kérdezd atyádat, és ő megtanít, * vagy őseidet, ők majd elmondják neked! ---

    8. Amikor a Mennybenlakó szétválasztotta a népeket, * amikor különválasztotta Ádám fiait, \\
    8. és kijelölte a népek határait, * Izraelt is, fiainak száma szerint: \\
    9. az Úr osztályrésze lett az ő népe, * Jákob az ő kimért öröksége. ---

    10. A pusztában találta meg őt, * a félelem helyén, a zord magányosság földjén. \\
    10. Körülvette, és tanítgatta, * és mint a szeme fényét, úgy őrizte. \\
    11. Mint fiókáit repülni tanító sasmadár, * amely fölöttük lebeg, \\
    11. kitárta szárnyát, őt meg fölemelte, * és tulajdon vállain vitte. \\
    12. Vezére az Úr volt egyedül, * és idegen isten nem volt vele.
}
\end{psalmsheet}

\redtext{3. antifóna}

\printsheet{QuamAdmirabileEst}

\begin{psalmsheet}[tone=1, continuous, number={8. zsoltár}, title={Isten fönsége és az ember méltósága}, motto={Mindent lába alá vetett, őt magát meg az egész Egyház fejévé tette (Ef 1, 22).}]
  \psalmtext{
    2. Uram, mi Urunk, * a te neved széles e világon mily csodálatos! ---

    2. Az egeknél fenségesebb * a te dicsőséged! \\
    3. Kisdedek és csecsemők ajkán † zendítesz dicséretet ellenségeiddel szemben, * hogy elnémíts ellenséget és lázadót. ---

    4. Szemlélem az eget: ujjaid műve az; * a holdat és a csillagokat: te alkottad őket. \\
    5. Ugyan mi az ember, hogy törődsz vele, * az ember fia, hogy gondod van rá? ---

    6. Kevéssel tetted kisebbé az angyaloknál, † dicsőséggel és fénnyel koronáztad, * fölébe emelted kezed műveinek. \\
    7. Mindent az ő lába alá vetettél: * minden juhot, barmot, a mező vadjait, ---

    8. az ég madarait s a tenger halait, * amelyek a tenger ösvényeit járják. \\
    10. Uram, mi Urunk, * a te neved széles e világon mily csodálatos!
}
\end{psalmsheet}

\parttitle[Iz 1,16-18]{Rövid olvasmány}

Mosdjatok meg, s tisztuljatok meg! El gonosz tetteitekkel színem elől, ne tegyetek többé %
rossz\underline{a}t! † Tanuljatok meg jót tenni: keressétek az igazságot, segítsétek az %
elnyomottakat, szolgáltassatok igazságot az árvának, s védelmezzétek az özvegyet! %
Aztán gyertek, s szálljatok velem perbe! - m\underline{o}ndja az Úr. * Ha olyanok volnának is %
bűneitek, mint a skarlát, fehérek lesznek, mint a hó; és ha olyan vörösek is, mint a bíbor, %
olyanok lesznek, mint \underline{a} gy\underline{a}pjú.

\parttitle{Rövid válaszos ének}

\printsheet{IpseLiberabitMe}

\parttitle[Lk 1,68-79]{Benedictus}

\printsheet{VadamAdPatremMeum}

\commentary{Minden sor intiummal.}

\begin{psalmsheet}[tone=1, continuous, initium, title={Ének a Messiásról és Előfutáráról}]
  \psalmtext{
    1. (+) Áldott az Úr, atyáink Istene, * mert meglátogatta és megváltotta az ő népét; \\
    2. erős szabadítót támasztott minékünk * szolgájának, Dávidnak családjából. \\
    3. Amint szólott a szentek szájával, * ősidők óta a próféták ajka által, \\
    4. megszabadít az ellenség kezéből, * mindazoktól, akik gyűlölettel néznek minket; \\
    5. atyáinkkal irgalmat gyakorol, * hogy szent szövetségére emlékezzék, \\
    6. az esküre, amelyet Ábrahám atyánknak esküdött, * hogy nekünk váltja be, amit ígért; \\
    7. hogy félelem nélkül és megszabadulva az ellenség kezéből, * neki szolgálatot teljesítsünk: \\
    8. szentségben és igazságban járjunk előtte * napról napra, amíg élünk. ---

    9. Téged pedig, gyermek, † a fölséges Isten prófétájának fognak mondani, * mert az Úr előtt jársz, egyengetni az ő útját; \\
    10. az üdvösség ismeretére tanítod nemzetét, * hogy bocsánatot nyerjen minden bűnük \\
    11. Istenünk irgalmas szívétől, * amellyel meglátogat minket felkelő Napunk a magasságból, \\
    12. hogy fényt hozzon azoknak, akik sötétségben és halálos homályban ülnek, * lépteinket pedig a béke útjára vezérelje.
}
\end{psalmsheet}

\parttitle{Fohászok}

\begin{preces}{%
  Mindig és mindenütt adjunk hálát Krisztus Urunknak, aki üdvözít bennünket, és kérjük bizalommal:
}{Urunk, siess segítségünkre kegyelmeddel!}
  Add, hogy tisztán meg tudjuk őrizni testünket,

  \reddash és így lakást vehessen benne a Szentlélek!

  Taníts arra, hogy már kora reggeltől testvéreink szolgálatára szenteljük magunkat,

  \reddash és egész nap mindenben teljesítsük akaratodat!

  Add, hogy az örök életre megmaradó kenyeret keressük,

  \reddash amelyet te adsz nekünk!

  Járjon közben értünk Édesanyád, a bűnösök menedéke,

  \reddash hogy jóságosan megbocsásd bűneinket!
\end{preces}
Mi Atyánk...

\parttitle{Könyörgés}

Istenünk, te már ezen a földön mennyei javakban részesítesz minket, amikor szentségeiddel %
belekapcsolsz mennyei életedbe. Kérünk, kormányozz minket földi utainkon, és vezess el örök %
világosságodra. A mi Urunk, Jézus Krisztus, a te Fiad által, aki veled él és uralkodik a %
Szentlélekkel egységben, Isten mindörökkön-örökké.

\end{document}
